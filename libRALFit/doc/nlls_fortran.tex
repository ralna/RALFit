% Copyright (c) 2016, The Science and Technology Facilities Council (STFC)
% All rights reserved.
\documentclass{spec}

\usepackage{amsfonts,bm,amsmath}
\usepackage{verbatim}
\usepackage{algorithm, algpseudocode}
\usepackage{caption}
\usepackage{calc}


% Copyright (c) 2016, The Science and Technology Facilities Council (STFC)
% All rights reserved.
% set the release and package names
\newcommand{\libraryname}{RAL}
\newcommand{\packagename}{NLLS}
\newcommand{\fullpackagename}{\libraryname\_\packagename}
\newcommand{\versionum}{0.5.0}
\newcommand{\versiondate}{15 January 2016}
\newcommand{\version}{\versionum}
\newcommand{\vx}{ {\bm x} } % macro for a vector x
\newcommand{\vr}{ {\bm r} } % macro for a vector r
\newcommand{\vs}{ {\bm s} } % macro for a vector s
\newcommand{\vt}{ {\bm t} } % macro for a vector t
\newcommand{\vg}{ {\bm g} } % macro for a vector g
\newcommand{\ve}{ {\bm e} } % macro for a vector d
\newcommand{\vd}{ {\bm d} } % macro for a vector d
\newcommand{\vp}{ {\bm p} } % macro for a vector p
\newcommand{\vw}{ {\bm w} } % macro for a vector w
\newcommand{\vH}{ {\bm H} } % macro for a matrix H
\newcommand{\vD}{ {\bm D} } % macro for a matrix D
\newcommand{\vhess}{ {\bm H_f} } % macro for a matrix H
\newcommand{\vJ}{ {\bm J} } % macro for a matrix J
\newcommand{\tx}{ {\tt x} } % macro for a vector x
\newcommand{\tr}{ {\tt r} } % macro for a vector r
\newcommand{\tg}{ {\tt g} } % macro for a vector g
\newcommand{\td}{ {\tt d} } % macro for a vector d
\newcommand{\ty}{ {\tt y} } % macro for a vector d
\newcommand{\tH}{ {\tt H} } % macro for a matrix H
\newcommand{\vI}{ {\bm I} } % macro for a matrix I
\newcommand{\vW}{ {\bm W} } % macro for a matrix W
\newcommand{\thess}{ {\tt Hf} } % macro for a matrix H
\newcommand{\tJ}{ {\tt J} } % macro for a matrix J
\newcommand{\iter}[2][k]{ #2_{#1}^{}} % macro for an iteration
\newcommand{\comp}[2][i]{ #2_{#1}^{}} % macro for a component of a vector

% data types for the derived types
\newcommand{\scalarinteger}[1]{\itt{#1} is a scalar of type {\tt INTEGER} }
\newcommand{\scalarlogical}[1]{\itt{#1} is a scalar of type {\tt LOGICAL} }
\newcommand{\scalarcharacter}[2]{
  \itt{#1} is a scalar of type {\tt CHARACTER} and length {\tt #2}
}
\newcommand{\scalarreal}[1]{\itt{#1} is a scalar of package type}
\newcommand{\ronearrayinteger}[1]{is a rank-1 array of type {\tt INTEGER} and size {\tt #1} }
\newcommand{\ronearrayreal}[2]{\itt{#1} is a rank-1 array of package type and size {\tt #2}\ }
% data types for the calling sequences
\newcommand{\scalarintegerio}{is an \intentout scalar of type {\tt INTEGER} }
\newcommand{\scalarintegerii}{is an \intentin scalar of type {\tt INTEGER} }

\newcommand{\ronearrayrealii}[1]{is an \intentin rank-1 array of package type and size {\tt #1} }
\newcommand{\ronearrayrealiiopt}[1]{is an optional \intentin rank-1 array of package type and size {\tt #1} }
\newcommand{\ronearrayrealio}[1]{is an \intentout rank-1 array of package type and size {\tt #1} }
% Fortran calling routine
\newcommand{\mainroutine}{{\tt nlls\_solve}}
\newcommand{\onestep}{{\tt nlls\_iterate}}
% Fortran customization
\newcommand{\ct}{\%}
\newcommand{\fortranonly}[1]{#1}
\newcommand{\conly}[1]{}
\newcommand{\vecentry}[2]{\texttt{#1(#2)}}
\begin{document}

\hslheader

\hslsummary

% Copyright (c) 2016, The Science and Technology Facilities Council (STFC)
% All rights reserved.
% The beginning summary section
{\tt \fullpackagename} computes a solution $\vx$ to the non-linear least-squares problem
\begin{equation}
\min_\vx \  F(\vx) := \frac{1}{2}\| \vr(\vx) \|_{\vW}^2 + \frac{\sigma}{p}\| \vx\|_2^p,
\label{eq:nlls_problem}
\end{equation}
where $\vW\in\mathbb{R}^{m\times m}$ is a diagonal, non-negative, weighting matrix, and $\vr(\vx) =(\comp[1]{r}(\vx), \comp[2]{r}(\vx),...,\comp[m]{r}(\vx))^T$ is a non-linear function.

A typical use may be to fit a function $f(\vx)$ to the data $y_i, \ t_i$, weighted by the uncertainty of the data, $\sigma_i$, so that
$$r_i(\vx) := y_i - f(\vx;t_i),$$
and $\vW$ is the diagonal matrix such that $\vW_{ii} = (1/\sqrt{\sigma_i}).$  For this reason
we refer to the function $\vr$ as the \emph{residual} function.
% the fit of the data $y$ to some non-linear function ${\bm f} : \mathbb{R}^n \rightarrow \mathbb{R}^m$
% ($m>n$).
% The $n$ variables that are fitted are $\vx=(x_1,x_2,...,x_n)^T$.
% \textcolor{blue}{Some confusion: the $y_i$ don't appear again.}

The algorithm is iterative.
At each point, $\iter{\vx}$, the algorithm builds a model of the function at the next step, $F({\iter{\vx}+\iter{\vs}})$, which we refer to as $m_k(\cdot)$.  We allow either a Gauss-Newton model, a (quasi-)Newton model, or a Newton-tensor model; see Section \ref{sec:model_description} for more details about these models.

Once the model has been formed we find a candidate for the next step by either solving a trust-region sub-problem of the form
\begin{equation}
\iter{\vs} = \arg \min_{\vs} \ \iter{m} (\vs) \quad \mathrm{s.t.} \quad  \|\vs\|_B \leq \Delta_k,\label{eq:tr_subproblem}
\end{equation}
or by solving the regularized problem
\begin{equation}
\iter{\vs} = \arg \min_{\vs} \ \iter{m} (\vs)  + \frac{1}{\Delta_k}\cdot \frac{1}{p} \|\vs\|_B^p,\label{eq:reg_subproblem}
\end{equation}
where $\Delta_k$ is a parameter of the algorithm (the trust region radius or the inverse of the regularization parameter respectively), $p$ is a given integer, and $B$ is a symmetric positive definite weighting matrix that is calculated by the algorithm.
The quantity
\[\rho = \frac{F(\iter{\vx}) - F(\iter{\vx} + \iter{\vs})}{\iter{m}(\iter{\vx}) - \iter{m}(\iter{\vx} + \iter{\vs})}\]
is then calculated.
If this is sufficiently large we accept the step, and $\iter[k+1]{\vx}$ is set to $\iter{\vx} + \iter{\vs}$; if not, the parameter $\Delta_k$
is reduced and  the resulting new trust-region sub-problem is solved.  If the step is very successful -- in that $\rho$ is close to one --
$\Delta_k$ is increased.

This process continues until either the residual, $\|\vr(\iter{\vx})\|_\vW$, or a measure of the gradient,
$\|{\iter{\vJ}}^T\vW\vr(\iter{\vx})\|_2 / \|\vr(\iter{\vx})\|_\vW$, is sufficiently small.


%%% Local Variables:
%%% mode: latex
%%% TeX-master: "nlls_fortran"
%%% End:


%!!!!!!!!!!!!!!!!!!!!!!!!!!!!
\hslattributes
\hslversions{\versionum\ (\versiondate)}.
\hslIRDCZ Real (single, double).
\hsllanguage Fortran 2003 subset (F95+TR155581).
\hsldate January 2016.
\hslorigin The Numerical Analysis Group, Rutherford Appleton Laboratory.
\hslremark The development of this package was
partially supported by EPSRC grant EP/M025179/1.

%!!!!!!!!!!!!!!!!!!!!!!!!!!!
\newpage
\hslhowto

\subsection{Calling sequences}

Access to the package requires a {\tt USE} statement \\ \\
\indent\hspace{5mm}{\tt use ral\_nlls\_double}
% \noindent
% If it is required to use more than one module at the same time, the derived
% types  (see Section~\ref{derived types})
% must be renamed in one of the {\tt USE} statements.
\medskip

\noindent The user can then call one of the procedures:
\vspace{-0.1cm}
\begin{description}
\item {\tt NLLS\_SOLVE}  solves the non-linear least squares problem (\ref{eq:nlls_problem}).
\item {\tt NLLS\_ITERATE} performs one iteration for the non-linear least squares solver.
\end{description}

%%%%%%%%%%%%%%%%%%%%%% derived types %%%%%%%%%%%%%%%%%%%%%%%%

\hsltypes
\label{derived types}
For each problem, the user must employ the derived types defined by the
module to declare scalars of the types {\tt NLLS\_inform} and
{\tt NLLS\_options}.  If {\tt nlls\_iterate} is to be used, then
a scalar of the type {\tt NLLS\_workspace} must also be defined.
The following pseudocode illustrates this.
\begin{verbatim}
      use nlls_module
      ...
      type (NLLS_inform) :: inform
      type (NLLS_options) :: options
      type (NLLS_workspace) :: work ! needed if nlls_iterate to be called
      ...
\end{verbatim}
The components of {\tt NLLS\_options} and {\tt NLLS\_inform} are explained
in Sections~\ref{typeoptions} and \ref{typeinform}.


%%%%%%%%%%%%%%%%%%%%%% argument lists %%%%%%%%%%%%%%%%%%%%%%%%
\hslarguments
\subsubsection{Optional arguments}\label{Optional arguments}

We use square brackets {\tt [ ]} to indicate \optional\ arguments.
In each
call, optional arguments follow the argument {\tt inform}.  Since we
reserve the right to add additional optional arguments in future
releases of the code, {\bf we strongly recommend that all optional
arguments be called by keyword, not by position}.

\subsubsection{Integer and package types}
%{\tt INTEGER} denotes default {\tt INTEGER} and
%{\tt INTEGER(long)} denotes {\tt INTEGER(kind=selected\_int\_kind(18))}.
The term {\bf package type} is used to mean
default real if the single precision version is being used and
double precision real for the double precision version.

\subsubsection{To solve the non-linear least squares problem}
\label{sec:solve}

To solve the non-linear least squares problem a call of the following form should be made:

\begin{verbatim}
   call nlls_solve(n, m, X, eval_r, eval_J, eval_Hf, params, options, inform[, weights])
\end{verbatim}

\begin{description}
\itt{n} \scalarintegerii that holds the number $n$ of
variables to be fitted; i.e., $n$ is the length of the unknown vector $\bm x$.
\textbf{Restriction:} \texttt{n}$>${\tt 0}.

\itt{m} \scalarintegerii that holds the number $m$ of
data points available; i.e., $m$ is the number of residuals $r_i$.
\textbf{Restriction:} \texttt{m}$\geq$\texttt{n}

\itt{X} is an \intentinout\  rank-1 array of package type
and size {\tt n}.  On entry, it must hold the initial guess for $\bm x$, and on
successful exit it holds the solution to the non-linear least squares problem.

\itt{eval\_r} is a {\tt PROCEDURE} that, given a point $\iter{\vx}$, returns the vector $\vr(\iter{\vx})$.
Further details of the format required are given in Section~\ref{sec::function_eval}.

\itt{eval\_J} is a {\tt PROCEDURE} that, given a point $\iter{\vx}$,
returns the $m \times n$ Jacobian matrix, $\iter{\vJ}$, of $\vr$ evaluated at $\iter{\vx}$.
Further details of the format required are given in Section~\ref{sec::function_eval}.

\itt{eval\_Hf} is a {\tt PROCEDURE} that, given vectors $\vx \in \mathbb{R}^n$
and $\vr \in \mathbb{R}^m$, returns the quantity $\sum_{i=1}^m (\vr)_i \nabla^2 \vr_i (\vx)$.
Further details of the format required are given in Section~\ref{sec::function_eval}.  If {\tt NLLS\_options\%exact\_second\_derivative = .false.}, then this is not referenced.

\itt{params} is an \intentin\ scalar of class {\tt params\_base\_type} that holds parameters to
be passed to the user-defined routines {\tt eval\_r}, {\tt eval\_J}, and {\tt eval\_Hf}.
Further details of its use are given in Section~\ref{sec::function_eval}.

\itt{options}is an \intentin\  scalar  of type {\tt NLLS\_options}
(see Section~\ref{typeoptions}).

\itt{inform} is an \intentinout\ scalar of type
{\tt NLLS\_inform}. Its components provide information about the execution
of the subroutine, as explained in Section~\ref{typeinform}.

\itt{weights} \ronearrayrealiiopt{m}.  If present, {\tt weights} holds the square-roots of the
diagonal entries of the weighting matrix, $\vW$, in (\ref{eq:nlls_problem}).  If absent, then the norm in (\ref{eq:nlls_problem}) is taken to be the 2-norm, that is, $\vW = I$.

\end{description}

\subsection{To iterate once}
\label{sec:iterate}
Alternatively, the user may step through the solution process one iteration at
a time by making a call of the following form:

\begin{verbatim}
   call nlls_iterate(n, m, X, w, eval_F, eval_J, eval_HF, params, options, inform[, weights])
\end{verbatim}

\begin{description}

\item[\texttt{n}, \texttt{m}, \texttt{eval\_F}, \texttt{eval\_J}, \texttt{eval\_HF}, \texttt{params}, \texttt{info} and \texttt{options}] are as described in Section~\ref{sec:solve}.

\itt{X} is an \intentinout\  rank-1 array of package type
and size {\tt n}.  On the first call, it must hold the initial guess for
$\bm x$. On return it holds the value of $\bm x$ at the current iterate, and
must be passed unaltered to any subsequent call to \texttt{nlls\_iterate()}.

\itt{w} is an \intentinout\ scalar of type \texttt{NLLS\_workspace}. It is used
to store the current state of the iteration and should not be altered by the
user.

\end{description}

The user may use the components \texttt{info\%convergence\_normf} and
\texttt{info\%convergence\_normg} to determine whether the iteration has
converged.

\subsection{User-supplied function evaluation routines}
\label{sec::function_eval}
To evaluate the residual, Jacobian and Hessian at a point, the user
must supply subroutines that perform this operation that the package
{\tt ral\_nlls} will call internally.

In order to pass user-defined data into the evaluation calls, {\tt params\_base\_type} is extended to a {\tt user\_type}, as follows:
\begin{verbatim}
   type, extends( params_base_type ) :: user_type
      ! code declaring components of user_type
   end type user_type
\end{verbatim}
We recommend this type is wrapped in a module with the user-defined routines
for evaluating the function, Jacobian, and Hessian.

The components of the extended type are accessed through a \texttt{select type}
construct:
\begin{verbatim}
   select type(params)
   type is(user_type)
     ! code that accesses components of params that were defined within user_type
   end select
\end{verbatim}

\subsubsection{For evaluating the function $\vr(\vx)$}

A subroutine must be supplied to calculate $\vr(\vx)$ for a given vector $\vx$.
It must implement the following interface:

\begin{verbatim}
abstract interface
   subroutine eval_r(n, m, params, x, r, status)
      integer, intent(in) :: n
      integer, intent(in) :: m
      class(params_base_type), intent(in) :: params
      double precision, dimension(n), intent(in) :: x
      double precision, dimension(m), intent(out) :: r
      integer, intent(inout) :: status
   end subroutine eval_r
end interface
\end{verbatim}

% Copyright (c) 2016, The Science and Technology Facilities Council (STFC)
% All rights reserved.
% description of eval_r

\begin{description}
   \itt{n, m, params} are passed unchanged as provided in the call to
   \mainroutine/\onestep.

   \itt{x} holds the current point $\iter{\vx}$ at which to evaluate $\vr(\iter{\vx})$.

   \itt{r} must be set by the routine to hold the residual function
      evaluated at the current point $\iter{\vx}$, $\vr(\iter{\vx})$.

   \fortranonly{
   \itt{status} is initialised to \texttt{0} before the routine is
      called. If it is set to a non-zero value by the routine, then \newline
      \mainroutine /\onestep \
      will exit with an error.
    }
\end{description}
\conly{
\textbf{Return value:} The function should return \texttt{0} on success. A non-zero return value will cause the least squares fitting to abort with an error code.
}

%%% Local Variables:
%%% mode: latex
%%% TeX-master: "nlls_fortran"
%%% End:


\subsubsection{For evaluating the function $\vJ = \nabla \vr(\vx)$}
A subroutine must be supplied to calculate $\vJ = \nabla \vr(\vx)$ for a given vector $\vx$. It
must implement the following interface:

\begin{verbatim}
abstract interface
   subroutine eval_J(n, m, params, x, J, status)
      integer, intent(in) :: n
      integer, intent(in) :: m
      class(params_base_type), intent(in) :: params
      double precision, dimension(n), intent(in)  :: x
      double precision, dimension(n*m), intent(out) :: J
      integer, intent(inout) :: status
  end subroutine eval_J
end interface
\end{verbatim}

% Copyright (c) 2016, The Science and Technology Facilities Council (STFC)
% All rights reserved.
% subroutine for eval_j


\begin{description}
   \itt{n, m, params} are passed unchanged as provided in the call to
      \mainroutine/\onestep.

   \itt{x} holds the current point $\iter{\vx}$ at which to evaluate
      $\vJ(\iter{\vx})$.

   \itt{J} must be set by the routine to hold the Jacobian of the residual
      function evaluated at the current point $\iter{\vx}$, $\vr(\iter{\vx})$.
      \vecentry{J}{i*m+j} must be set to hold $\nabla_{x_j} r_i(\iter{\vx})$.

   \fortranonly{
     \itt{status} is initialised to \texttt{0} before the routine is
     called. If it is set to a non-zero value by the routine, then
     \mainroutine/\onestep \ will exit with an error.
   }
\end{description}
\conly{
  \textbf{Return value:} The function should return \texttt{0} on success. A
non-zero return value will cause the least squares fitting to abort with an
error code.
}
%%% Local Variables:
%%% mode: latex
%%% TeX-master: "nlls_fortran"
%%% End:


\subsubsection{For evaluating the function $Hf = \sum_{i=1}^m r_i(\vx) \vW \nabla^2 r_i(\vx)$}
A subroutine must be supplied to calculate $Hf = \sum_{i=1}^m (\vr)_i \nabla^2 r_i(\vx)$ for given vectors $\vx \in \mathbb{R}^n$ and $\vr \in \mathbb{R}^m$; here \((\vr)_i\) denotes the $i$th component of the vector $\vr$. The subroutine must implement the following interface:

\begin{verbatim}
abstract interface
   subroutine eval_Hf_type(n, m, params, x, r, Hf, status)
       integer, intent(in) :: n
       integer, intent(in) :: m
       class(params_base_type), intent(in) :: params
       double precision, dimension(n), intent(in)  :: x
       double precision, dimension(m), intent(in)  :: r
       double precision, dimension(n*n), intent(out) :: Hf
       integer, intent(inout) :: status
     end subroutine eval_Hf_type
end interface
\end{verbatim}

% Copyright (c) 2016, The Science and Technology Facilities Council (STFC)
% All rights reserved.
% subroutine for eval_j


\begin{description}
  \itt{n, m, params} are passed unchanged as provided in the call to \mainroutine/\onestep.

  \itt{x} holds the current point $\iter{\vx}$ at which to evaluate $\sum_{i=1}^m (\vr)_i \nabla^2 r_i(\vx)$.

  \itt{r} holds $\vW\vr(\vx)$, the (weighted) residual, as computed from vector returned by the last call to \texttt{eval\_r}.

  \itt{Hf} must be set by the routine to hold the matrix $\sum_{i = 1}^m (\vr)_{i}\nabla^2\comp{r}(\iter{\vx})$,
  held by columns as a vector, where $(\vr)_i$ denotes the $i$th component of  $\texttt{r}$, the vector passed to the routine.

  \fortranonly{
     \itt{status} is initialised to \texttt{0} before the routine is
     called. If it is set to a non-zero value by the routine, then
     \mainroutine/\onestep \ will exit with an error.
   }
\end{description}
\conly{
  \textbf{Return value:} The function should return \texttt{0} on success. A
non-zero return value will cause the least squares fitting to abort with an
error code.
}
%%% Local Variables:
%%% mode: latex
%%% TeX-master: "nlls_fortran"
%%% End:


%%%%%%%%%%%%%%%%%%%%%%%%%%%%%%%%%%%%%%%%%%%%%%%%%%%%%%%%%%%%%%%%%%%%%%



\subsection{The options derived data type}
\label{typeoptions}


% Copyright (c) 2016, The Science and Technology Facilities Council (STFC)
% All rights reserved.
% the options derived data type

\fortranonly{
The derived data type {\tt NLLS\_options}
is used to hold controlling data. The components  are automatically
given default values in the definition of the type.
}
\conly{
The structure of type {\tt struct ral\_nlls\_options} is used to hold
controlling data. The components must be initialised through a call to
\texttt{ral\_nlls\_default\_options()}.
}

\vspace{2mm}

\noindent {\bf Components that control printing}
\begin{description}

\scalarinteger{out}
with default value {\tt 6} that is used as the \conly{Fortran }unit number for general messages. If it is negative, these messages will be suppressed.

\scalarinteger{print\_level}
with default value {\tt 0} that controls the level of output required.
\begin{description}
\item{\tt $\leq$ 0} No informational output will occur.
\item{\tt = 1} Gives a one-line summary for each iteration.
\item{\tt = 2} As 1, plus gives a summary of the inner iteration for each iteration.
\item{\tt > 2} As 2, and gives increasingly verbose (debugging) output.
\end{description}
The default is {\tt print\_level} $=$ 0.
\end{description}

\scalarlogical{print_options}
with default value {\tt false}, determines whether to print a list of all options and their values at the beggining of the solve.

\scalarinteger{print_header}
with default value {\tt 30}, it will print the column header every 30 iterations when {\tt print\_level} is greater than 1.

\noindent {\bf Components that control the main iteration}.

\begin{description}

\scalarinteger{maxit}
of iterations the algorithm is allowed to take before being stopped.  The default value is {\tt 100}.  This is not accessed if {\tt nlls\_iterate} is used.

\scalarinteger{model}
that specifies the model, $m_k(\cdot)$, used.  Possible values are
\begin{description}
  \item{\tt 1} Gauss-Newton (no Hessian).
  \item{\tt 2} (Quasi-)Newton (uses exact Hessian if {\tt exact\_second\_derivatives} is true, otherwise builds an approximation to the Hessian).
%  \item{\tt 3} Barely second-order (Hessian matrix approximated by the identity).
  \item{\tt 3} Hybrid method (mixture of Gauss-Newton/(Quasi-)Newton, as determined by the package).
  \item{\tt 4} Tensor-Newton method.
\end{description}
The default is {\tt model = 3}.

\scalarinteger{type\_of\_method}
that specifies the type of globalization method used.  Possible values are
\begin{description}
  \item{\tt 1} Trust-region method.
  \item{\tt 2} Regularization.
\end{description}
The default is {\tt type\_of\_method = 1}.

\scalarinteger{nlls\_method}
that specifies the method used to solve (or approximate the solution to) the trust-region sub problem.  Possible values are
\begin{description}
  \item{\tt 1} Powell's dogleg method (approximates the solution).
  \item{\tt 2} The Adachi-Iwata-Nakatsukasa-Takeda (AINT) method.
  \item{\tt 3} The More-Sorensen method.
  \item{\tt 4} {\sc Galahad}'s {\sc DTRS} method if {\tt type\_of\_method=1}, or {\sc Galahad}'s {\sc DRQS} method if {\tt type\_of\_method=2}.
\end{description}
If {\tt type\_of\_method = 4}, then only {\tt nlls\_method = 4} is permitted.
The default is {\tt nlls\_method = 4}.

\scalarreal{stop\_g\_absolute}
that specifies the absolute tolerance used in the convergence test on \(\|{\iter{\vJ}}^T\vr(\iter{\vx}))\|/\|\vr(\iter{\vx})\|\).
The default value is {\tt stop\_g\_absolute = 1e{-5}}.

\scalarreal{stop\_g\_relative}
that specifies the relative tolerance used in the convergence test on \(\|{\iter{\vJ}}^T\vr(\iter{\vx})\|/\|\vr(\iter{\vx})\|\).
The default value is {\tt stop\_g\_relative = 1e{-8}}.

\scalarreal{stop\_f\_absolute}
that specifies the absolute tolerance used in the convergence test on \(\|\vr(\iter{\vx})\|\).
The default value is {\tt stop\_f\_absolute = 1e{-8}}.

\scalarreal{stop\_f\_relative}
that specifies the relative tolerance used in the convergence test on \(\|\vr(\iter{\vx})\|\).
The default value is {\tt stop\_f\_relative = 1e{-8}}.

\scalarreal{stop\_s}
that specifies the tolerance used in the convergence test on \(\|\iter{\vs}\|\).
The default value is {\tt stop\_s = eps}.

\scalarinteger{relative\_tr\_radius}
that specifies whether the initial trust region radius should be scaled.
The default value is {\tt relative\_tr\_radius = 0}.

\scalarreal{initial\_radius\_scale}
that specifies the scaling parameter for the initial trust region radius, which is only used if {\tt relative\_tr\_radius = 1}.
The default value is {\tt initial\_radius\_scale = 1.0}.

\scalarreal{initial\_radius}
that specifies the initial trust-region radius, $\Delta$.
The default value is {\tt initial\_radius = 100.0}.

\scalarreal{regularization\_weight}
that specifies the regularization weight, $\sigma$, used in the least-squares problem (\ref{eq:nlls_problem}).
The default value is {\tt regularization\_weight = 0.0}.

\scalarreal{regularization\_power}
that specifies the regularization weight, $p$, used in the least-squares problem (\ref{eq:nlls_problem}).
The default value is {\tt regularization\_power = 0.0}.


\scalarreal{maximum\_radius}
that specifies the maximum size permitted for the trust-region radius.
The default value is {\tt maximum\_radius = 1e8}.

\scalarreal{eta\_successful}
that specifies the smallest value of $\rho$ such that the step is accepted.
The default value is {\tt eta\_successful = 1e-8}.

% success_but_reduce is also available, but not documented

\scalarreal{eta\_very\_successful}
that specifies the value of $\rho$ after which the trust-region radius is increased.
The default value is {\tt eta\_very\_successful = 0.9}.

\scalarreal{eta\_too\_successful}
that specifies that value of $\rho$ after which the step is accepted, but keep the trust-region radius unchanged.
The default value is {\tt eta\_too\_successful = 2.0}.

\scalarreal{radius\_increase}
that specifies the factor to increase the trust-region radius by.
The default value is {\tt radius\_increase = 2.0}.

\scalarreal{radius\_reduce}
that specifies the factor to decrease the trust-region radius by.
The default value is {\tt radius\_reduce = 0.5}.

\scalarinteger{tr\_update\_strategy}
that specifies the strategy used to update $\Delta_k$.  Possible values are:
\begin{description}
\item {\tt 1} use the usual step function.
\item {\tt 2} use a the continuous method.
\end{description}
See Section~\ref{sec:step_accept} for more details.
The default value is {\tt 1}.

\scalarreal{hybrid\_switch}
that specifies the value, if {\tt model = 3}, at which second derivatives are used
The default value is {\tt hybrid\_switch = 0.1}.

\scalarlogical{exact\_second\_derivatives}
that, if {\tt true}, signifies that the exact second derivatives are available (and, if {\tt false}, approximates them using a secant method).
The default value is {\tt false}.

\scalarinteger{scale}
that specifies how, if at all, we scale the Jacobian.  We calculate a diagonal scaling matrix, ${\tt D}$, as follows:
\begin{description}
\item{\tt 0} No scaling is applied ${\tt D} ={\tt I}$.
\item{\tt = 1} Scale using the rows of the Jacobian, \({\tt D}_{i,i} = \|{\tt J}(i,:)\|_2^2\).
\item{\tt = 2} Scale using the rows of the Hessian,  \({\tt D}_{i,i} = \|{\tt H}(i,:)\|_2^2\), where ${\tt H}$ is the (approximation to) the Hessian matrix.
\end{description}
The default value is {\tt scale = 1}

\scalarlogical{scale\_trim\_max}
that specifies whether or not to trim large values of the scaling matrix, ${\tt D}$.
If {\tt true}, ${\tt D}_{i,i} \leftarrow min({\tt D}_{i,i}, {\tt scale\_max})$.
The default value is {\tt true}.

\scalarreal{scale\_max}
that specifies the maximum value allowed if {\tt scale\_trim\_max = true}.
The default value is {\tt 1e11}.

\scalarlogical{scale\_trim\_min}
that specifies whether or not to trim small values of the scaling matrix, ${\tt D}$.
If {\tt true}, ${\tt D}_{i,i} \leftarrow max({\tt D}_{i,i}, {\tt scale\_max})$.
The default value is {\tt true}.

\scalarreal{scale\_min}
that specifies the minimum value allowed if {\tt scale\_trim\_max = true}.
The default value is {\tt 1e-11}.

\scalarlogical{scale\_require\_increase}
that specifies whether or not to require ${\tt D}_{i,i}$ to increase before updating it.
The default value is {\tt false}.

\scalarlogical{calculate\_svd\_J}
that specifies whether or not to calculate the singular value decomposition of ${\tt J}$
at each iteration.
The default value is {\tt false}.

% \scalarlogical{setup\_workspaces}
% that specifies whether or not to setup the workspaces.  If {\tt setup\_workspaces = false}, then the user must pass a pre-assigned scalar of type {\tt NLLS\_workspace} (generated with the appropriate options) to {\tt nlls\_iterate}, which must come either from a previous call to {\tt nlls\_iterate}.

% \scalarlogical{remove\_workspaces}

\scalarinteger{more\_sorensen\_maxits}
that, if {\tt nlls\_method = 3}, specifies the maximum number of iterations allowed in the More-Sorensen method.
The default value is {\tt more\_sorensen\_maxits = 500}.

\scalarreal{more\_sorensen\_shift}
that, if {\tt nlls\_method = 3}, specifies the shift to be used in the More-Sorensen method.
The default value is {\tt more\_sorensen\_shift = 1e-13}.

\scalarreal{more\_sorensen\_tiny}
that, if {\tt nlls\_method = 3}, specifies the value below which numbers are considered to be essentially zero.
The default value is {\tt more\_sorensen\_tiny = 10.0 * eps}.

\scalarreal{more\_sorensen\_tol}
that, if {\tt nlls\_method = 3}, specifies the tolerance to be used in the More-Sorensen method.
The default value is {\tt more\_sorensen\_tol = 1e-3}.

\scalarreal{hybrid\_tol}
that, if \(\|{\iter{\vJ}}^T \vW \vr(\vx_k) \|_2 < \mathtt{hybrid\_tol} * 0.5 \|\vr(\vx_k)\|_\vW^2\), switches to a \newline(quasi-)Newton method.
The default value is {\tt hybrid\_tol = 2.0}.

\scalarinteger{hybrid\_switch\_its}
that sets how many iterates in a row must the condition in the definition of {\tt hybrid\_tol} hold before a switch.
The default value is {\tt hybrid\_switch\_its = 1}.

\scalarreal{reg\_order}
that, if {\tt nlls\_method = 2}, the order of the regularization used ($p$ in (\ref{eq:reg_subproblem})).  If {\tt reg\_order = 0.0}, then the algorithm
chooses an appropriate value of $p$. The default is {\tt reg\_order = 0.0}.

\scalarinteger{inner\_method}
that, if {\tt nlls\_method = 4}, specifies the method used to pass in
the regularization parameter to the inner non-linear least squares solver.
Possible values are:
\begin{description}
  \item {\tt 1} The current regularization parameter is passed in as a base regularization parameter.
  \item {\tt 2} The regularization term is added to the sum of squares to be solved in the inner problem.
\end{description}
The default is {\tt inner\_method = 2}.

\fortranonly{
\scalarlogical{output\_progress\_vectors}
that, if true, outputs the progress vectors {\tt nlls\_inform\%resvec} and {\tt nlls\_inform\%gradvec} at the end of the routine.
The default value is {\tt false}.
}

\end{description}


%%% Local Variables:
%%% mode: latex
%%% TeX-master: "nlls_fortran"
%%% End:


\subsection{The derived data type for holding information}
\label{typeinform}

% Copyright (c) 2016, The Science and Technology Facilities Council (STFC)
% All rights reserved.
\fortranonly{
The derived data type {\tt NLLS\_inform} is used to hold information from the execution of {\tt nlls\_solve} and {\tt nlls\_iterate}.
}
\conly{
The structure of type {\tt struct ral\_nlls\_inform} is used to hold information from the execution of {\tt ral\_nlls}.
}

The components are:
\begin{description}
\scalarinteger{status}
that gives the exit status of the subroutine.  See Section~\ref{hslerrors} for details.
\scalarcharacter{error\_message}{80} that holds the error message corresponding to the exit status.
\scalarinteger{alloc\_status} that gives the status of the last attempted allocation/deallocation.
\scalarcharacter{bad\_alloc}{80} that holds the name of the array that was being allocated when an error was flagged.
\scalarinteger{iter} that gives the total number of iterations performed.
\scalarinteger{f\_eval} that gives the total number of evaluations of the objective function.
\scalarinteger{g\_eval} that gives the total number of evaluations of the gradient of the objective function.
\scalarinteger{h\_eval} that gives the total number of evaluations of the Hessian of the objective function.
\scalarinteger{convergence\_normf} that tells us if the test on the size of \(\vr\) is satisfied.
\scalarinteger{convergence\_normg} that tells us if the test on the size of the gradient is satisfied.
\scalarinteger{convergence\_norms} that tells us if the test on the step length is satisfied.
\ronearrayreal{resvec}{iter+1} that, if {\tt nlls\_options\ct output\_progress\_vectors=true}, holds the vector of residuals.
\ronearrayreal{gradvec}{iter+1} that, if {\tt nlls\_options\ct output\_progress\_vectors=true}, holds the vector of gradients.
\scalarreal{obj} that holds the value of the objective function at the best estimate of the solution determined by the algorithm.
\scalarreal{norm\_g} that holds the gradient of the objective function at the best estimate of the solution determined by the package.
\scalarreal{scaled\_g} that holds the gradient of the objective function at the best estimate of the solution determined by the package.
\scalarinteger{external\_return} that gives the error code that was returned by a call to an external routine.
\scalarcharacter{external\_name}{80} that holds the name of the external code that flagged an error.
\scalarreal{step} holds the size of the last step taken.
\end{description}


%%% Local Variables:
%%% mode: latex
%%% TeX-master: "nlls_fortran"
%%% End:


%%%%%%%%%%%%%%%%%%%%%% Warning and error messages %%%%%%%%%%%%%%%%%%%%%%%%

\hslerrors

A successful return from a subroutine in the package is indicated by
{\tt NLLS\_inform\%status} having the value zero.
A non-zero value is associated with an error message that by default will
be output on unit {\tt NLLS\_options\%error}.

% Copyright (c) 2016, The Science and Technology Facilities Council (STFC)
% All rights reserved.
% Error and warning messages -- same for C and Fortran

Possible values are:
\begin{description}
\item{} {\tt -1} Maximum number of iterations reached without convergence.
\item{} {\tt -2} Error from evaluating a function/Jacobian/Hessian.
\item{} {\tt -3} Unsupported choice of model.
\item{} {\tt -4} Error return from an external routine.
\item{} {\tt -5} Unsupported choice of method.
\item{} {\tt -6} Allocation error.
\item{} {\tt -7} Maximum number of reductions of the trust radius reached.
\item{} {\tt -8} No progress being made in the solution.
\item{} {\tt -9} \texttt{n}$>$\texttt{m}.
\item{} {\tt -10} Unsupported trust region update strategy.
\item{} {\tt -11} Unable to valid step when solving trust region subproblem.
\item{} {\tt -12} Unsupported scaling method.
\item{} {\tt -13} Error accessing pre-allocated workspace.
\item{} {\tt -101} Unsupported model in dogleg (\texttt{nlls\_method = 1}).
\item{} {\tt -201}  All eigenvalues are imaginary (\texttt{nlls\_method=2}).
\item{} {\tt -202} Matrix with odd number of columns sent to \texttt{max\_eig} subroutine (\texttt{nlls\_method=2}).
\item{} {\tt -301} {\tt nlls\_options\ct more\_sorensen\_max\_its} is exceeded in \texttt{more\_sorensen} subroutine (\texttt{nlls\_method=3}).
\item{} {\tt -302} Too many shifts taken in \texttt{more\_sorensen} subroutine (\texttt{nlls\_method=3}).
\item{} {\tt -303} No progress being made in \texttt{more\_sorensen} subroutine (\texttt{nlls\_method=3}).
\item{} {\tt -401} {\tt nlls\_options\ct model = 4} selected, but {\tt nlls\_options\ct exact\_second\_derivatives} is set to {\tt false}.
\item{} {\tt -501} {\tt nlls\_options\ct type\_of\_method = 2} selected, but {\tt nlls\_options\ct type\_of\_method} is not equal to {\tt 4}.
\end{description}


\hslgeneral


\hslio
   Error messages on unit {\tt NLLS\_options\%error} and warning
   and diagnostic messages on unit \newline{\tt NLLS\_options\%out},
   respectively. These have default value {\tt 6};
   printing of these messages is suppressed if the relevant unit number
   is negative or if {\tt NLLS\_options\%print\_level} is negative.
\hslrestrictions {\tt m$\ge$n$\ge$1}.

\hslmethod
\label{method}

% Copyright (c) 2016, The Science and Technology Facilities Council (STFC)
% All rights reserved.
\label{sec:Method}
% describe the method we use

Algorithm~\ref{alg:nlls_solve} describes the method used to minimize the cost function
$F(\vx)$ in equation (\ref{eq:nlls_problem}). This is an iterative method that, at each iteration, calculates and returns a step $\vs$ that reduces the model by an acceptable amount by solving (or approximating a solution to) either the trust-region subproblem (\ref{eq:tr_subproblem}) or a regularized problem (\ref{eq:reg_subproblem}).

\begin{algorithm}
\caption{nlls\_solve}
\label{alg:nlls_solve}
  \begin{algorithmic}[1]
    \State  {\tt {\bf function} \tx }$=${\tt nlls\_solve}$(\iter[0]{\tx},\text{\tt options}{\tt[,W]})$
    \If {${\tt W}$ not present}
    \State ${\tt W=I}$
    \EndIf
    \State $\sigma = ${\tt options\ct regularization\_weight}, $p = ${\tt options\ct regularization\_power}
    \State $\iter[0]{\tr} =  {\tt W * }${\tt eval\_r}$(\iter[0]{\tx})$, $\iter[0]{\tJ} = {\tt W *}$ {\tt eval\_J}$(\iter[0]{\tx})$
    \Comment Evaluate residual and Jacobian at initial guess
    \State $\Delta = ${\tt options\ct initial\_radius}
    \State $ \iter[0]{\tg} = - {\iter[0]{\tJ}}^T\iter[0]{\tr} - \sigma \|\iter[0]{\tx}\|^{p-2}\iter[0]{\tx}$
    \State $\tt normF_0 = 0.5\|\iter[0]{\tr}\|^2 + \frac{\sigma}{p} \|\iter[0]{\tx}\|^p$
    \If {{\tt options\ct model == 1}}
    \Comment Gauss-Newton model
    \State $\iter[0]{\thess} = {\tt 0}$
    \State {\tt use\_second\_derivatives = false}
    \ElsIf {{\tt options\ct model == 2}}
    \Comment (Quasi-)Newton
    \State $\iter[0]{\thess} = ${\tt eval\_HF}${\tt (\iter[0]{\tx},W * \iter[0]{\tr})}$
    \State {\tt use\_second\_derivatives = true}
    \ElsIf {{\tt options\ct model == 3}}
    \Comment Hybrid algorithm
    \State {\tt hybrid\_tol = options\ct hybrid\_tol * }
    ${\tt (\| \iter[0]{\tg} \| / normF_0 )}$
    \State $\iter[0]{\thess} = {\tt 0}$
    \Comment Use first-order information only initially
    \State {\tt use\_second\_derivatives = false}
    \State ${\iter[temp]{\thess}} = {\tt 0}$
    \Comment Build up a Hessian in parallel when Gauss-Newton used
    \ElsIf{{\tt options\ct model == 4}}
    \Comment Use the Newton-Tensor model
    \State $\iter[0]{\thess} = ${\tt eval\_HF}${\tt (\iter[0]{\tx},W * \iter[0]{\tr})}$
    \If{$p \ne 0$ and $p \ne 2$}
    \State $\iter[0]{\thess}  = \iter[0]{\thess}  + \sigma \|\iter[0]{\tx}\|^{p-2}
    \left(I + \frac{{\iter[0]{\tx}}{\iter[0]{\tx}}^T}{\|\iter[0]{\tx}\|^2}\right)$
    \EndIf
    \State {\tt use\_second\_derivatives = true}
    \EndIf
    \For { $k = {\tt 0}, \dots, \text{\tt options\ct maxit}$}
      \While{ ${\tt success} \ne  1$ }
        \State ${\td}$ = \Call{{\tt calculate\_step}}{{$\tt
            \iter[k]{\tJ}, \iter[k]{\tr}, \iter[k]{\thess},\iter[k]{\tg},\Delta$}}
        \Comment Calculate a potential step $\td$
        \State $\iter[k+1]{\tx} = \iter[k]{\tx} + \td$
        \State $\iter[k+1]{\tr} = {\tt W * \text{\tt eval\_r}}(\iter[k]{\tx})$
        \State ${\tt normF_{k+1}}  = 0.5\|\iter[k+1]{\tr}\|^2 + \frac{\sigma}{p} \|\iter[k+1]{\tx}\|^p$
        \Comment Evaluate the residual at the new point
        \State $\rho = \tt (normF_{k+1} - normF_k)/(m_k(0) - m_k(\td)) $
        \Comment If model is good, $\rho$ should be close to one
          \If{ ${\tt \rho } >$ {\tt control\ct eta\_successful}}
          \State ${\tt success} = 1$
        \EndIf
        \State ${\tt \Delta = }${\tt update\_trust\_region\_radius}${\tt (\Delta,\rho)}$
      \EndWhile
      \State $\iter[k+1]{\tJ} = {\tt W * \text{\tt eval\_J}}(\iter[k+1]{\tx})$
      \Comment Evaluate the Jacobian at the new point
      \State $\iter[k+1]{\tg} = -{\iter[k+1]{\tJ}}^T\iter[k+1]{\tr}- \sigma \|\iter[k+1]{\tx}\|^{p-2}\iter[k+1]{\tx}$
      \If {{\tt \text{\tt options\ct model} == 3}}
        \If {{\tt use\_second\_derivatives}}
          \If { $\|\iter[k+1]{\tg}\| > \|\iter[k]{\tg} \| $}
          \State {\tt use\_second\_derivatives = false}
          \Comment Switch back to Gauss-Newton
          \State ${\iter[temp]{\thess}} = \iter[k]{\thess}$, $\iter[k]{\thess} = 0$
          \Comment Copy Hessian back to temp array
          \EndIf
        \Else
      \algstore{myalg}
  \end{algorithmic}

\end{algorithm}

\begin{algorithm}
  \ContinuedFloat
  \begin{algorithmic}
    \algrestore{myalg}

          \If { $ \tt \|\iter[k+1]{\tg}\| / normF_{k+1} < \text{\tt hybrid\_tol}$}
          \State {\tt hybrid\_count = hybrid\_count + 1}
          \Comment Update the number of steps in a row this has failed
          \If {{\tt hybrid\_count == options\ct hybrid\_count\_switch\_its}}
            \State {\tt use\_second\_derivatives = true}x
            \State {\tt hybrid\_count = 0}
            \State ${\iter[temp]{\thess}} = {\iter[k]{\thess}}$
            \Comment Copy approximate Hessian back
          \EndIf
          \EndIf
        \EndIf
      \If {{\tt ({\bf not} use\_second\_derivatives) {\bf and} ({\bf not} options\ct exact\_second\_derivatives) } }
      \State ${\iter[temp]{\thess}} = \Call{{\tt rank\_one\_update}}{\td ,\iter[k]{\tg},\iter[k+1]{\tg}, \iter[k+1]{\tr},\iter[k]{\tJ},\iter[temp]{\thess}}$
      \EndIf
    \EndIf

    \If { {\tt use\_second\_derivatives} }
      \If { {\tt options\ct exact\_second\_derivatives} }
        \State $\iter[k+1]{\thess} = {\tt \text{\tt eval\_HF}(\iter[0]{\tx},W\iter[0]{\tr})}$
      \Else
        \State ${\iter[k+1]{\thess}} = \Call{{\tt rank\_one\_update}}{\td ,\iter[k]{\tg},\iter[k+1]{\tg}, \iter[k+1]{\tr},\iter[k]{\tJ},\iter[k]{\thess}}$
      \EndIf
    \EndIf
    \If{$p \ne 0$ and $p \ne 2$}
    \State $\iter[k+1]{\thess}  = \iter[k+1]{\thess}  + \sigma \|\iter[k+1]{\tx}\|^{p-2}
    \left(I + \frac{{\iter[k+1]{\tx}}{\iter[k+1]{\tx}}^T}{\|\iter[k+1]{\tx}\|^2}\right)$
    \EndIf

    \If {$ \tt\|\iter[k+1]{\tr}\| < max(${\tt options\ct stop\_g\_absolute}, {\tt options\ct stop\_g\_relative}$* \|\iter[k]{\tr}\|)$}
    \State {\tt return}
    \Comment converged due to residual being small
    \ElsIf{$ \tt\frac{\|\iter[k+1]{\tg}\|}{\|\iter[k+1]{\tr}\|} < max( \text{\tt options\ct stop\_g\_absolute}, \text{\tt options\ct stop\_g\_relative} * \left(\frac{\|\iter[0]{\tg}\|}{\|\iter[0]{\tr}\|}\right))$}
    \State {\tt return}
    \Comment converged due to gradient being small
    \EndIf
    \EndFor
  \end{algorithmic}
\end{algorithm}

The subroutine \texttt{nlls\_iterate} performs one iteration of the algorithm
\texttt{nlls\_solve}, allowing the user greater control over stopping and/or monitoring the progress of the algorithm.

\subsection{Incorporating the regularization term}
\label{sec:reg_problem}

If a non-zero regularization term is required in (\ref{eq:nlls_problem}), then this is handled by transforming the problem internally into a least squares problem.  The
formulation used will depend on the value of $p$.

{\bf If $\bf p = 2$}, we solve a least squares problem with $n$ additional degrees of freedom.
The new function, $\widehat{\vr} : \mathbb{R}^{n}\rightarrow\mathbb{R}^{m+n}$, takes $\widehat{\vr}_i(\vx) = \vr_i(\vx)$, for $i = 1,\dots, m$, and $\widehat{\vr}_{m+j}(\vx) =
\sqrt{\sigma}[\vx]_j$ for $j = 1,\dots, n$, where $[\vx]_j$ denotes the $j$th component of $\vx$.  We therefore have that $\nabla \widehat{\vr}_{m+j}(\vx) = \sqrt{\sigma}\ve^j$ (where $[\ve^j]_i = \delta_{ij}$), and the second derivatives vanish.

{\bf If $\bf p \ne 2$}, then we solve a least squares problem
 with one additional degree of freedom.  In this case the new function, $\widehat{\vr} : \mathbb{R}^{n}\rightarrow\mathbb{R}^{m+1}$, again takes $\widehat{\vr}_i(\vx) = \vr_i(\vx)$, for $i = 1,\dots, m$, but now $\widehat{\vr}_{m+1}(\vx) = \left(\frac{2\sigma}{p}\right)^{\frac{1}{2}}\|\vx\|^{\frac{p}{2}}.$  We therefore have that
$\nabla \widehat{\vr}_{m+1}(\vx) = \left(\frac{2\sigma}{p}\right)^{\frac{1}{2}}\|\vx\|^{\frac{p-4}{2}}\vx^T$.
The second derivative is given by %on the value of $p$:
\(
\nabla^2\widehat{\vr}_{m+1} =
%\begin{cases}
%   \hspace{1.5cm}\sigma^{\frac{1}{2}} I & \text{if } p = 2 \\
   \left(\frac{2\sigma}{p}\right)^{\frac{1}{2}}\|\vx\|^{\frac{p-4}{2}}\left(I + \frac{\vx\vx^T}{\|\vx\|^2}\right).% & \text{otherwise}
%\end{cases}.
\)


Either problem can be solved implictly from the un-regularized problem by updating the relevant quantities in the algorithm, as shown in Algorithm~\ref{alg:nlls_solve}.

\subsection{The models}
\label{sec:model_description}

A vital component of the algorithm is the choice of model employed.  There are four choices
available, controlled by the parameter {\tt  nlls\_method\ct model}.

\begin{description}
  \item {\tt options\ct model = 1}: this implements the {\bf Gauss-Newton} model.  Here we replace $\vr(\iter[k]{\vx} + \vs)$ by its first-order Taylor approximation, $\vr(\iter{\vx}) + \iter{\vJ}\vs$. The model is therefore given by    \begin{equation}
m_k^{GN}(\vs) = \frac{1}{2} \|\vr(\iter{\vx}) + \iter{\vJ}\vs\|_\vW^2.
\label{eq:gauss-newton-model}
\end{equation}
\item {\tt options\ct model = 2}: this implements the {\bf Newton} model.
Here, instead of approximating the residual, $\vr(\cdot)$, we take as our model the second-order Taylor approximation of the function, $F(\iter[k+1]{\vx}).$  Namely, we use
\begin{equation}
  \label{eq:newton-model}
  m_k^{N}(\vs) = F(\iter{\vx}) + {\iter{\vg}}^T\vs + \frac{1}{2}\vs^T\left( {\iter{\vJ}}^T \vW \iter{\vJ} + \iter{\vH}\right) \vs,
\end{equation}
where $\iter{\vg} = {\iter{\vJ}}^T\vW \vr(\iter{\vx})$ and $\iter{\vH} = \sum_{i=1}^m\iter[i]{r}(\iter{\vx}) \vW \nabla^2 \iter[i]{r}(\iter{\vx}).$
Note that $m_k^{N}(\vs) = m_k^{GN}(\vs) + \frac{1}{2}\vs^T\iter{\vH} \vs$.

If the second derivatives of $\vr(\cdot)$ are not available
(i.e., {\tt options\ct exact\_second\_derivatives = false}),
then the method approximates the matrix $\iter{\vH}$ using the method of Dennis, Gay, and Welsch [4, Chapter 10];
see Algorithm~\ref{alg:rank_one_update}.


\item {\tt options\ct model = 3}: this implements a {\bf hybrid} model.
In practice the Gauss-Newton model tends to work well far away from the solution, whereas
Newton performs better once we are near to the minimum (particularly if the residual is
large at the solution).
This option will try to switch between these two models, picking the model that is most appropriate for the step.
In particular, we start using $m_k^{GN}(\cdot)$,
and switch to $m_k^{N}(\cdot)$ if $\|{\iter{\vg}}\|_2 \leq ${\tt options\ct hybrid\_tol}$\frac{1}{2}\|\vr(\iter{\vx})\|^2_\vW$ for more than {\tt options\ct hybrid\_switch\_its} iterations in a row. If, in subsequent iterations, we fail to get
a decrease in the function value, then the algorithm interprets this as being not sufficiently close to the solution, and thus switches back to using the Gauss-Newton model.

\item {\tt options\ct model = 4}: this implements a {\bf Newton-tensor} model.
This uses a second order Taylor approximation to the residual, namely
\[r_i(\iter{\vx} + \vs) \approx (\iter{\vt}(\vs))_i := r_i(\iter{\vx}) + (\iter{\vJ})_i\vs + \frac{1}{2}\vs^T B_{ik}\vs,\]
where $(\iter{\vJ})_i$ is the {\tt i}th row of $\iter{\vJ}$, and $B_{ik}$ is $\nabla^2 r_i(\iter{\vx})$.  We use this to define our model
\begin{equation}
  \label{eq:newton-tensor-model}
  m_k^{NT}(\vs) = \frac{1}{2}\|\vt_k(\vs)\|_\vW^2.
\end{equation}

\end{description}


\begin{algorithm}
\caption{{\tt rank\_one\_update}}
\label{alg:rank_one_update}
  \begin{algorithmic}
    \State {\bf function} $\iter[k+1]{\thess} = $ \Call{{\tt rank\_one\_update}}{$\td ,\iter[k]{\tg},\iter[k+1]{\tg}, \iter[k+1]{\tr},\iter[k]{\tJ},\iter[k]{\thess}$}
    \State $\ty = \iter[k]{\tg} - \iter[k+1]{\tg}$ \State
    $\widehat{\ty} = {\iter[k]{\tJ}}^T \iter[k+1]{\tr} -
    \iter[k+1]{\tg}$ \State $\widehat{\iter[k]{\thess}} = \min\left(
      1, \frac{|\td^T\widehat{\ty}|}{|\td^T\iter[k]{\thess}\td|}\right)
    \iter[k]{\thess}$ \State $\iter[k+1]{\thess} =
    \widehat{\iter[k]{\thess}} + \left(({\iter[k+1]{\widehat{\ty}}} -
      \iter[k]{\thess}\td )^T\td\right)/\ty^T\td$

  \end{algorithmic}
\end{algorithm}


\subsection{The subproblem solves}
\label{sec:subproblem solves}


The main algorithm (Algorithm \ref{alg:nlls_solve}) calls a number of subroutines.
The most vital is the subroutine {\tt calculate\_step}, which finds a step that
minimizes the model chosen, subject to a globalization strategy.  The algorithm
supports the use of two such strategies: using a trust-region, and regularization.
If Gauss-Newton, (quasi-)Newton, or a hybrid method is used
({\tt options\ct model = 1,2,3}), then the model function is quadratic, and
the methods available to solve the subproblem are described in
Sections \ref{sec:trust-region} and \ref{sec:regularization}.
If the Newton-Tensor model is selected ({\tt options\ct model = 4}), then this
model is not quadratic, and the methods available are described in Section~\ref{sec:newton_tensor_subproblem}.

Note that, when calculating the step, if the initial regularization parameter $\sigma$ in (\ref{eq:nlls_problem}) is
non-zero, then we must modify ${\iter[k]{\tJ}}^T\iter[k]{\tJ}$  to take into
account the Jacobian of the modified least squares problem being solved.  Practically, this amounts to making the change
\[
{\iter[k]{\tJ}}^T\iter[k]{\tJ} = {\iter[k]{\tJ}}^T\iter[k]{\tJ} +
 \begin{cases}
   \sigma I & \text{if }p = 2\\
   \frac{\sigma p}{2} \|\iter[k]{\vx}\|^{p-4}\iter[k]{\vx}{\iter[k]{\vx}}^T & \text{otherwise}
 \end{cases}.
\]

\subsubsection{The trust region method ({\tt options\ct type\_of\_method = 1})}
\label{sec:trust-region}

If {\tt options\ct type\_of\_method = 1}, then a trust-region method is used.  Such a method solves the subproblem (\ref{eq:tr_subproblem}), and we take as our next step
the minimum of the model within some radius of the current point.  The method used to solve
this is dependent on the control parameter {\tt options\ct nlls\_method}. The algorithms called for each of the options are listed below:
\begin{description}
\item {\tt options\ct nlls\_method = 1}: this approximates the solution to (\ref{eq:tr_subproblem}) by using Powell's dogleg method.  This takes as the step a linear combination of the Gauss-Newton step and the steepest descent step, and the method used is described in Algorithm \ref{alg:dogleg}.
\item {\tt options\ct nlls\_method = 2}: this solves the trust region subproblem using the trust region solver of  Adachi, Iwata, Nakatsukasa, and Takeda.  This reformulates the
problem (\ref{eq:tr_subproblem}) as a generalized eigenvalue problem, and solves that.  See
[1] for more details.
\item {\tt options\ct nlls\_method = 3}: this solves the trust region subproblem using
a variant of the More-Sorensen method.  In particular, we implement Algorithm 7.3.6
 in Trust Region Methods by Conn, Gould and Toint [2].
\item {\tt options\ct nlls\_method = 4}: this solves the trust region subproblem by first
converting the problem into the form
$$\min_\vp \vw^T \vp + \frac{1}{2} \vp^T \vD \vp \quad {\rm s.t.} \quad \|\vp\| \leq \Delta,$$
where $\vD$ is a diagonal matrix.  We do this by performing an eigen-decomposition of
the Hessian in the model.  Then, we call the {\sc Galahad} routine {\sc DTRS}; see
the {\sc Galahad} [3] documentation for further details.
\end{description}

\begin{algorithm}
\caption{dogleg}
\label{alg:dogleg}
  \begin{algorithmic}[1]
    \State {\bf function} \Call{{\tt dogleg}}{{$\tt
            \tJ, {\tr}, \thess, \tg,\Delta$}}
        \State $\alpha = \|\tg\|^2 / \|\tJ * \tg\|^2$
        \State $\td_{\rm sd} = \alpha \,\tg$
        \State Solve $\td_{\rm gn} = \arg \min_{\tx}\|\tJ \tx- \tr\|_2$
        \If {$\|\td_{\rm gn}\| \leq \Delta$}
        \State $\td = \td_{\rm gn}$
        \ElsIf {$\|\alpha \, \td_{\rm sd}\| \geq \Delta$}
        \State $\td = (\Delta / \|\td_{\rm sd}\|) \td_{\rm sd}$
        \Else
        \State $\td = \alpha \, \td_{\rm sd} + \beta\, (\td_{\rm gn} - \alpha \td_{\rm sd})$, where $\beta$ is chosen such that $\|\td\| = \Delta$
        \EndIf
  \end{algorithmic}
\end{algorithm}

\subsubsection{Regularization ({\tt options\ct type\_of\_method = 2})}
\label{sec:regularization}

If {\tt options\ct type\_of\_method = 2}, then the next step is taken to be the minimum
of the model with a regularization term added (\ref{eq:reg_subproblem}).  At present,
only one method of solving this subproblem is supported:

\begin{description}
\item {\tt options\ct nlls\_method = 4}: this solves the regularized subproblem by first
converting the problem into the form
$$\min_\vp \vw^T \vp + \frac{1}{2} \vp^T \vD \vp + \frac{1}{p}\|\vp\|_2^p,$$
where $\vD$ is a diagonal matrix.  We do this by performing an eigen-decomposition of
the Hessian in the model.  Then, we call the {\sc Galahad} routine {\sc DRQS}; see
the {\sc Galahad} [3] documentation for further details.
\end{description}

\subsubsection{Newton-Tensor subproblem}
\label{sec:newton_tensor_subproblem}

If {\tt options\ct model = 4}, then the non-quadratic Newton-Tensor model is used.  As such, none of the established subproblem solvers described in Section~\ref{sec:trust-region} or Section~\ref{sec:regularization} can be used.

If we use regularization (with $p=2$), then the subproblem we need to solve is of the form
\begin{equation}
\min_\vs \frac{1}{2}\sum_{i=1}^mW_{ii}{(\vt_k(\vs))_i}^2 + \frac{1}{2\Delta_k}\|\vs\|_2^2
\label{eq:reg_newton_tensor_subproblem}
\end{equation}
Note that (\ref{eq:reg_newton_tensor_subproblem}) is a sum-of-squares,
and as such can be solved by calling {\tt ral\_nlls} recursively.
We support two options:
\begin{description}
  \item {\tt options\ct inner\_method = 1}:
    if this option is selected, then {\tt nlls\_solve} is called to
    solve (\ref{eq:newton-tensor-model}) directly.  The current regularization parameter
    of the `outer' method is used as a base regularization in the `inner' method,
    so that the (quadratic) subproblem being solved in the `inner' call is of the form
    \[ \min_\vs \, m_k(\vs) + \frac{1}{2}\left(\frac{1}{\Delta_k} + \frac{1}{\delta_k}\right)\|\vs\|_B^2, \]
    where $m_k(\vs)$ is a quadratic model of (\ref{eq:newton-tensor-model}), $\Delta_k$ is the
    (fixed) regularization parameter of the outer iteration, and $\delta_k$ the regularization
    parameter of the inner iteration, which is free to be updated as required by the method.

  \item {\tt options\ct inner\_method = 2}: in this case we use {\tt ral\_nlls} to solve
    the regularized model (\ref{eq:reg_newton_tensor_subproblem}) directly.
    The number of parameters for this subproblem is $n+m$.  Specifically, we have a
    problem of the form
    \[
    \min_\vs \frac{1}{2} \|\widehat{\vr}(\vs)\|_\vW^2,
    \quad \text{where }
    (\widehat{\vr}(\vs))_i =
    \begin{cases}
      (\vt_k(\vs))_i &  1 \leq i \leq m \\
      \frac{1}{\sqrt{\Delta_k}}s_i& m+1 \leq i \leq n+m
    \end{cases}.
    \]
    This subproblem can then be solved using any of the methods described in
    Sections~\ref{sec:trust-region} or \ref{sec:regularization}.
\end{description}


\subsection{Accepting the step and updating the parameter}
\label{sec:step_accept}


Once a step has been suggested, we must decide whether or not to accept the step, and whether the trust region radius or regularization parameter, as appropriate, should grow, shrink, or remain the same.

These decisions are made with reference to a parameter, $\rho$, which measures the
ratio of the actual reduction in the model to the predicted reduction in the model.
If this is larger than {\tt options\ct eta\_successful}, then the is step accepted (see Line 28 of Algorithm~\ref{alg:nlls_solve}).

The value of $\Delta_k$ then needs to be updated, if appropriate.
The package supports two options: if
\begin{description}
\item  {\tt options\ct tr\_update\_strategy = 1:} in this case a step-function is used to decide whether or not to increase or decrease $\Delta_k$.
\item {\tt options\ct tr\_update\_strategy = 2}, then a continuous function is used to
  make the decision.
\end{description}
The method used is outlined in Algorithm~\ref{alg:update_tr}.

\begin{algorithm}
\caption{update\_trust\_region}
\label{alg:update_tr}
\begin{algorithmic}[1]
  \State {\bf function} $\Delta = $ \Call{{\tt update\_trust\_region\_radius}}{{$\Delta, \rho$}}
    \If {{\tt options\ct tr\_update\_strategy == 1}}
      \If {{$\tt \rho \leq \text{\tt options\ct eta\_success\_but\_reduce}$}}
      \State $\tt \Delta = \text{\tt options\ct radius\_reduce} * \Delta$
      \Comment reduce $\Delta$
      \ElsIf{{$\tt \rho \leq  \text{\tt options\ct eta\_very\_successful}$}}
      \State $\tt \Delta = \Delta$
      \Comment $\Delta$ stays unchanged
      \ElsIf{{$\tt \rho \leq \text{\tt options\ct eta\_too\_successful}$}}
      \State $\tt \Delta = \text{\tt options\ct radius\_increase} * \Delta$
      \Comment increase $\Delta$
      \ElsIf{{$\tt \rho > \text{\tt options\ct eta\_too\_successful}$}}
      \State $\tt \Delta = \Delta$
      \Comment too successful: accept step, but don't change $\Delta$
      \EndIf
    \ElsIf{{\tt options\ct tr\_update\_strategy == 2}}
    \State [on first call, set $\nu = 2.0$]
      \If{{$\tt \rho \geq \text{\tt options\ct eta\_too\_successful}$}}
        \State $\Delta = \Delta$
        \Comment $\Delta$ stays unchanged
      \ElsIf{{$\tt \rho > \text{\tt options\ct eta\_successful}$}}
        \State $\tt \Delta = \Delta * \min\left(\text{\tt options\ct radius\_increase},
          1 - \left( (\text{\tt options\ct radius\_increase} -1)*((1 - 2*\rho)^3)  \right)\right)$
        \State $\tt \nu = \text{\tt options\ct radius\_reduce}$
      \ElsIf{{$\tt \rho \leq \text{\tt options\ct eta\_successful}$}}
        \State $ \Delta = \nu * \Delta$
        \State $ \nu = 0.5 * \nu$
      \EndIf
    \EndIf
  \end{algorithmic}
\end{algorithm}



\hslreferences\\
$[1]$ Adachi, Satoru and Iwata, Satoru and Nakatsukasa, Yuji and Takeda, Akiko (2015).
Solving the trust region subproblem by a generalized eigenvalue problem.
Technical report, Mathematical Engineering, The University of Tokyo.\\
$[2]$ Conn, A. R., Gould, N. I., \& Toint, P. L. (2000). Trust region methods. SIAM.\\
$[3]$ Gould, N. I., Orban, D., \& Toint, P. L. (2003). GALAHAD, a library of thread-safe Fortran 90 packages for large-scale nonlinear optimization. ACM Transactions on Mathematical Software (TOMS), 29(4), 353-372.\\
$[4]$ Nocedal, J., \& Wright, S. (2006). Numerical optimization. Springer Science \& Business Media.

%%% Local Variables:
%%% mode: latex
%%% TeX-master: "nlls_fortran"
%%% End:
% todo!

\hslexample

% Copyright (c) 2016, The Science and Technology Facilities Council (STFC)
% All rights reserved.
% setup the example problem

Consider fitting the function $y(t) = x_1e^{x_2 t}$ to data $(\bm{t}, \bm{y})$
using a non-linear least squares fit.\\
The residual function is given by
$$
   r_i(\vx)  = x_1 e^{x_2 t_i} - y_i.
$$
We can calculate the Jacobian and Hessian of the residual as
$$
   \nabla r_i(\vx) = \left(\begin{array}{cc}
      e^{x_2 t_i} &
      t_i x_1 e^{x_2 t_i}
      \end{array}\right),
$$
$$
   \nabla^2 r_i(\vx) = \left(\begin{array}{cc}
      0                 & t_i e^{x_2 t_i}    \\
      t_i e^{x_2 t_i}     & x_1 t_i^2 e^{x_2 t_i}
   \end{array}\right).
$$

For some data points, $y_i$, $t_i$, $(i = 1,\dots,m)$ the user must return
$$  \vr(\vx) = \begin{bmatrix}
      r_1(\vx)\\
      \vdots \\
      r_m(\vx)
    \end{bmatrix}, \quad   \vJ(\vx) =
    \begin{bmatrix}
      \nabla r_1(\vx) \\
      \vdots \\
      \nabla r_m(\vx) \\
    \end{bmatrix}, \quad
    Hf(\vx) =
    \sum_{i=1}^m
    (\vr)_i \nabla^2 r_i(\vx),
$$
where, in the case of the Hessian, $(\vr)_i$ is the $i$th component of a residual vector passed to the user.

Given the data
\begin{center}
   \begin{tabular}{l|*{5}{r}}
      $i$   & 1 & 2 & 3  & 4  & 5 \\
      \hline
      $t_i$ & 1 & 2 & 4  & 5  & 8 \\
      $y_i$ & 3 & 4 & 6 & 11 & 20
   \end{tabular}
\end{center}
and initial guess $\vx = (2.5, 0.25)$, the following code performs the fit (with no
weightings, i.e., $\vW = \vI$).

%%% Local Variables:
%%% mode: latex
%%% TeX-master: "nlls_fortran"
%%% End:

\verbatiminput{../example/Fortran/nlls_example.f90}


\end{document}
