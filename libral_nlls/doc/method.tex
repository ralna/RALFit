% describe the method we use

The algorithm uses a trust-region method to minimize the cost function
$F(\vx)$, as in equation (\ref{eq:nlls_problem}).  The algorithm used by \texttt{nlls\_solve} is given as Algorithm~\ref{alg:nlls_solve}.  This is a trust region method that, at each iteration, calculates and returns step $\vd$ that reduces the model by an acceptable amount by solving (or approximating a solution to) the trust-region subproblem (\ref{eq:tr_subproblem}). 

\begin{algorithm}
\caption{nlls\_solve}
\label{alg:nlls_solve}
  \begin{algorithmic}[1]
    \State  {\tt {\bf function} \tx }$=${\tt nlls\_solve}$(\iter[0]{\tx},\text{\tt options}{\tt[,W]})$
    \If {${\tt W}$ not present}
    \State ${\tt W=I}$
    \EndIf
    \State $\iter[0]{\tr} =  {\tt W * }${\tt eval\_r}$(\iter[0]{\tx})$, $\iter[0]{\tJ} = {\tt W *}$ {\tt eval\_J}$(\iter[0]{\tx})$    
    \Comment Evaluate residual and Jacobian at initial guess
    \State $\Delta = ${\tt options\ct initial\_radius}
    \State $ \iter[0]{\tg} = - {\iter[0]{\tJ}}^T\iter[0]{\tr}$
    \If {{\tt options\ct model == 1}}
    \Comment Gauss-Newton model 
    \State $\iter[0]{\thess} = {\tt 0}$
    \State {\tt use\_second\_derivatives = false}
    \ElsIf {{\tt options\ct model == 2}}
    \Comment (Quasi-)Newton 
    \State $\iter[0]{\thess} = ${\tt eval\_HF}${\tt (\iter[0]{\tx},W * \iter[0]{\tr})}$
    \State {\tt use\_second\_derivatives = true}
    \ElsIf {{\tt options\ct model = 9}}
    \Comment Hybrid algorithm
    \State {\tt hybrid\_tol = options\ct hybrid\_tol * }${\tt (\| \iter[0]{\tg} \| / (0.5 * \|\iter[0]{\tr}\|^2) )}$
    \State $\iter[0]{\thess} = {\tt 0}$
    \Comment Use first-order information only initially
    \State {\tt use\_second\_derivatives = false}
    \State ${\iter[temp]{\thess}} = {\tt 0}$
    \Comment Build up a Hessian in parallel when Gauss-Newton used
    \EndIf
    \For { $k = {\tt 0}, \dots, \text{\tt options\ct maxit}$}
      \While{ ${\tt success} \ne  1$ } 
        \State ${\td}$ = \Call{{\tt calculate\_step}}{{$\tt 
            \iter[k]{\tJ}, \iter[k]{\tr}, \iter[k]{\thess},\iter[k]{\tg},\Delta$}}
        \Comment Calculate a potential step $\td$
        \State $\iter[k+1]{\tx} = \iter[k]{\tx} + \td$
        \State $\iter[k+1]{\tr} = {\tt W * \text{\tt eval\_r}}(\iter[k]{\tx})$
        \Comment Evaluate the residual at the new point
        \State $\rho = \tt 0.5 (\| \iter[k+1]{\tr} \|^2 - \| \iter[k]{\tr}\|^2)/(m_k(0) - m_k(\td)) $ 
        \Comment If model is good, $\rho$ should be close to one
          \If{ ${\tt \rho } >$ {\tt control\ct eta\_successful}}
          \State ${\tt success} = 1$
        \EndIf
        \State ${\tt \Delta = }${\tt update\_trust\_region\_radius}${\tt (\Delta,\rho)}$
      \EndWhile
      \State $\iter[k+1]{\tJ} = {\tt W * \text{\tt eval\_J}}(\iter[k+1]{\tx})$
      \Comment Evaluate the Jacobian at the new point
      \State $\iter[k+1]{\tg} = -{\iter[k+1]{\tJ}}^T\iter[k+1]{\tr}$
      \If {{\tt \text{\tt options\ct model} == 9}}
        \If {{\tt use\_second\_derivatives}}
          \If { $\|\iter[k+1]{\tg}\| > \|\iter[k]{\tg} \| $}
          \State {\tt use\_second\_derivatives = false}
          \Comment Switch back to Gauss-Newton
          \State ${\iter[temp]{\thess}} = \iter[k]{\thess}$, $\iter[k]{\thess} = 0$
          \Comment Copy Hessian back to temp array
          \EndIf
        \Else
          \If { $\|\iter[k+1]{\tg}\| / (0.5 * \|\iter[k+1]{\tr}\|^2 < \text{\tt hybrid\_tol})$}
          \State {\tt hybrid\_count = hybrid\_count + 1}
          \Comment Update the number of steps in a row this has failed
          \If {{\tt hybrid\_count == options\ct hybrid\_count\_switch\_its}}
            \State {\tt use\_second\_derivatives = true}
            \State {\tt hybrid\_count = 0}
            \State ${\iter[temp]{\thess}} = {\iter[k]{\thess}}$
            \Comment Copy approximate Hessian back
          \EndIf
          \EndIf
        \EndIf
      \algstore{myalg}
  \end{algorithmic}
  
\end{algorithm}

\begin{algorithm}
  \ContinuedFloat
  \begin{algorithmic}
    \algrestore{myalg}
      \If {{\tt ({\bf not} use\_second\_derivatives) {\bf and} ({\bf not} options\ct exact\_second\_derivatives) } }
      \State ${\iter[temp]{\thess}} = \Call{{\tt rank\_one\_update}}{\td ,\iter[k]{\tg},\iter[k+1]{\tg}, \iter[k+1]{\tr},\iter[k]{\tJ},\iter[temp]{\thess}}$
      \EndIf
    \EndIf

    \If { {\tt use\_second\_derivatives} } 
      \If { {\tt options\ct exact\_second\_derivatives} }
        \State $\iter[k+1]{\thess} = {\tt \text{\tt eval\_HF}(\iter[0]{\tx},W\iter[0]{\tr})}$
      \Else
        \State ${\iter[k+1]{\thess}} = \Call{{\tt rank\_one\_update}}{\td ,\iter[k]{\tg},\iter[k+1]{\tg}, \iter[k+1]{\tr},\iter[k]{\tJ},\iter[k]{\thess}}$
      \EndIf
    \EndIf
    \If {$ \tt\|\iter[k+1]{\tr}\| < max(${\tt options\ct stop\_g\_absolute}, {\tt options\ct stop\_g\_relative}$* \|\iter[k]{\tr}\|)$}
    \State {\tt return}
    \Comment converged due to residual being small
    \ElsIf{$ \tt\frac{\|\iter[k+1]{\tg}\|}{\|\iter[k+1]{\tr}\|} < max( \text{\tt options\ct stop\_g\_absolute}, \text{\tt options\ct stop\_g\_relative} * \left(\frac{\|\iter[0]{\tg}\|}{\|\iter[0]{\tr}\|}\right))$}
    \State {\tt return}
    \Comment converged due to gradient being small
    \EndIf
    \EndFor
  \end{algorithmic}
\end{algorithm}


The subroutine \texttt{nlls\_iterate} performs one iteration of the algorithm
\texttt{nlls\_solve}, allowing the user greater control over stopping and/or monitoring the progress of the algorithm.

The main algorithm (Algorithm \ref{alg:nlls_solve}) calls a number of subroutines.  
The most vital is the subroutine {\tt calculate\_step}, which solves the trust region 
subproblem (\ref{eq:tr_subproblem}) (or an approximation to it). The method used to solve
this is dependent on the control parameter {\tt options\ct nlls\_method}. The algorithms called for each of the options are listed below:
\begin{description}
\item {\tt options\ct nlls\_method = 1}: this approximates the solution to (\ref{eq:tr_subproblem}) by using Powell's dogleg method.  This takes as the step a linear combination of the Gauss-Newton step and the steepest descent step, and the method used is described in Algorithm \ref{alg:dogleg}.
\item {\tt options\ct nlls\_method = 2}: this solves the trust region subproblem using the trust region solver of  Adachi, Iwata, Nakatsukasa, and Takeda.  This reformulates the 
problem (\ref{eq:tr_subproblem}) as a generalized eigenvalue problem, and solves that.  See
[1] for more details.
\item {\tt options\ct nlls\_method = 3}: this solves the trust region subproblem using 
a variant of the More-Sorensen method.  In particular, we implement Algorithm 7.3.6
 in Trust Region Methods by Conn, Gould and Toint [2].
\item {\tt options\ct nlls\_method = 4}: this solves the trust region subproblem by first 
converting the problem into the form
$$\min_\vp \vw^T \vp + \frac{1}{2} \vp^T \vD \vp \quad {\rm s.t.} \quad \|\vp\| \leq \Delta,$$
where $\vD$ is a diagonal matrix.  We do this by performing an eigen-decomposition of 
the Hessian in the model.  Then, we call the {\sc Galahad} routine {\sc DTRS}; see 
the {\sc Galahad} [3] documentation for further details.
\end{description}

\begin{algorithm}
\caption{dogleg}
\label{alg:dogleg}
  \begin{algorithmic}[1]
    \State {\bf function} \Call{{\tt dogleg}}{{$\tt 
            \tJ, {\tr}, \thess, \tg,\Delta$}}
        \State $\alpha = \|\tg\|^2 / \|\tJ * \tg\|^2$
        \State $\td_{\rm sd} = \alpha \,\tg$
        \State Solve $\td_{\rm gn} = \arg \min_{\tx}\|\tJ \tx- \tr\|_2$
        \If {$\|\td_{\rm gn}\| \leq \Delta$}
        \State $\td = \td_{\rm gn}$
        \ElsIf {$\|\alpha \, \td_{\rm sd}\| \geq \Delta$}
        \State $\td = (\Delta / \|\td_{\rm sd}\|) \td_{\rm sd}$
        \Else
        \State $\td = \alpha \, \td_{\rm sd} + \beta\, (\td_{\rm gn} - \alpha \td_{\rm sd})$, where $\beta$ is chosen such that $\|\td\| = \Delta$
        \EndIf
  \end{algorithmic}
\end{algorithm}

Another vital component of the algorithm is the mechanism for updating the trust region radius.  The package supports two options: if {\tt options\ct tr\_update\_strategy = 1}, then 
a step-function is used to 
  decide whether or not to increase or decrease the trust region radius.  Alternatively, if
{\tt options\ct tr\_update\_strategy = 2}, then a continuous function is used to make the 
decision.  The method used is outlined in Algorithm~\ref{alg:update_tr}.

\begin{algorithm}
\caption{update\_trust\_region}
\label{alg:update_tr}
\begin{algorithmic}[1]    
  \State {\bf function} $\Delta = $ \Call{{\tt update\_trust\_region\_radius}}{{$\Delta, \rho$}}
    \If {{\tt options\ct tr\_update\_strategy == 1}}
      \If {{$\tt \rho \leq \text{\tt options\ct eta\_success\_but\_reduce}$}}
      \State $\tt \Delta = \text{\tt options\ct radius\_reduce} * \Delta$
      \Comment reduce $\Delta$
      \ElsIf{{$\tt \rho \leq  \text{\tt options\ct eta\_very\_successful}$}}
      \State $\tt \Delta = \Delta$
      \Comment $\Delta$ stays unchanged
      \ElsIf{{$\tt \rho \leq \text{\tt options\ct eta\_too\_successful}$}}
      \State $\tt \Delta = \text{\tt options\ct radius\_increase} * \Delta$
      \Comment increase $\Delta$
      \ElsIf{{$\tt \rho > \text{\tt options\ct eta\_too\_successful}$}}
      \State $\tt \Delta = \Delta$
      \Comment too successful: accept step, but don't change $\Delta$
      \EndIf
    \ElsIf{{\tt options\ct tr\_update\_strategy == 2}}
    \State [on first call, set $\nu = 2.0$]
      \If{{$\tt \rho \geq \text{\tt options\ct eta\_too\_successful}$}}
        \State $\Delta = \Delta$
        \Comment $\Delta$ stays unchanged
      \ElsIf{{$\tt \rho > \text{\tt options\ct eta\_successful}$}}
        \State $\tt \Delta = \Delta * \min\left(\text{\tt options\ct radius\_increase},  
          1 - \left( (\text{\tt options\ct radius\_increase} -1)*((1 - 2*\rho)^3)  \right)\right)$
        \State $\tt \nu = \text{\tt options\ct radius\_reduce}$
      \ElsIf{{$\tt \rho \leq \text{\tt options\ct eta\_successful}$}}
        \State $ \Delta = \nu * \Delta$
        \State $ \nu = 0.5 * \nu$
      \EndIf
    \EndIf
  \end{algorithmic}
\end{algorithm}

Finally, the application of a Quasi-Newton method is supported when exact second derivatives are not available.  The method of Dennis, Gay, and Welsch  is used, and implemented as described by Nocedal and Wright [4, Chapter 10]; see Algorithm~\ref{alg:rank_one_update}.

\begin{algorithm}
\caption{{\tt rank\_one\_update}}
\label{alg:rank_one_update}
  \begin{algorithmic}
    \State {\bf function} $\iter[k+1]{\thess} = $ \Call{{\tt rank\_one\_update}}{$\td ,\iter[k]{\tg},\iter[k+1]{\tg}, \iter[k+1]{\tr},\iter[k]{\tJ},\iter[k]{\thess}$}
    \State $\ty = \iter[k]{\tg} - \iter[k+1]{\tg}$ \State
    $\widehat{\ty} = {\iter[k]{\tJ}}^T \iter[k+1]{\tr} -
    \iter[k+1]{\tg}$ \State $\widehat{\iter[k]{\thess}} = \min\left(
      1, \frac{|\td^T\widehat{\ty}|}{|\td^T\iter[k]{\thess}\td|}\right)
    \iter[k]{\thess}$ \State $\iter[k+1]{\thess} =
    \widehat{\iter[k]{\thess}} + \left(({\iter[k+1]{\widehat{\ty}}} -
      \iter[k]{\thess}\td )^T\td\right)/\ty^T\td$

  \end{algorithmic}
\end{algorithm}

\hslreferences\\
$[1]$ Adachi, Satoru and Iwata, Satoru and Nakatsukasa, Yuji and Takeda, Akiko (2015).
Solving the trust region subproblem by a generalized eigenvalue problem.
Technical report, Mathematical Engineering, The University of Tokyo.\\
$[2]$ Conn, A. R., Gould, N. I., \& Toint, P. L. (2000). Trust region methods. SIAM.\\
$[3]$ Gould, N. I., Orban, D., \& Toint, P. L. (2003). GALAHAD, a library of thread-safe Fortran 90 packages for large-scale nonlinear optimization. ACM Transactions on Mathematical Software (TOMS), 29(4), 353-372.\\
$[4]$ Nocedal, J., \& Wright, S. (2006). Numerical optimization. Springer Science \& Business Media.

%%% Local Variables: 
%%% mode: latex
%%% TeX-master: "nlls_fortran"
%%% End: 